% !TEX root = ../main.tex
% !TEX spellcheck = en_US
\chapter{Background}
While RTS games has been around since the 80s\cite{adams06, rtsHistory}, only a handful of scientific articles can be found on teammate bots for RTS games, and current big RTS titles have yet to implement a good teammate bot. In general little research has been done for teammate bots throughout all genres; RTS researchers have focused on enemy bots to either create a fun opponent\cite{hagelback09} or to create the best bot to compete in RTS bot tournaments, such as  AIIDE's StarCraft AI Competition\cite{scaiide} and CIG's StarCraft AI Competition\cite{sccig}.

%: change the paragraph later
Next we will cover related research topics, beginning with teammate bots across all genres and asking what guidelines applies to RTS games; continuing with RTS enemy bots and asking, what implementation strategies exists, how a good bot shall play; and ending with communication between AI and humans including design choices to avoid player misinterpretation of the bot. After the research topics you will be updated with the current games across all genres using teammate bots and what current RTS games lack.

\section{Research}
\subsection{Teammate bots}
\label{sec:teammate_bots}
As mentioned, little research has been done in the area of teammate bots, especially for RTS games. To our knowledge there exist one paper on teammate bots\cite{mcgee10}; in their survey, McGee and Abraham, presents their definition of real-time teammate, which their survey is limited to. A summary of their definition reads; a real-time teammate bot
\begin{enumerate}
	\item works together with team players while taking into account the state, needs, behavior, goals, plans and intentions of these players;
	\item uses coordinated behaviors or decision-making\ldots
	\item {\ldots}that aligns with the team goals;
	\item where these coordinated behaviors or decision-making includes player-related uncertainty requiring inference, reasoning, classification, calculation, or another method; and
	\item whenever possible, prioritizing the player experience.
\end{enumerate}
We will use the same definition and follow these five points for our proposed bot.

In their survey\cite{mcgee10} McGee and Abraham noticed that, although human player participation and engagement are one key functionality of a game\cite{reynolds03}, often the player's preferences are neglected and the bot(s) behave what it thinks is the best for either just itself or both the player and itself; while the second option might sound as if it prioritizes the player, it does not, it steers the player into how to play rather than the player steering the bot. When the bot prioritizes the player, some challenges arise; how to create priority rules that do their job correctly\cite{mcgee10}, i.e. the bot has to know, or use a qualified guess, what the player wants. This is no easy task, probably impossible in the near future; humans have a hard time understanding each others intentions, why think AIs (that humans have created) understands us better\cite{norman07} without even asking?.

McGee and Abraham points out the lack of research on communication between human players and bots, “This survey suggests that there are also some aspects of real-time team-mate AI where there seems to be little or no work: ..., and communication.”\cite{mcgee10} Meaning “little or no work” spans through multiple topics and multiple genres; we have yet to find any paper that talks about communication between human players and bots, for any genre. Games in other genres has, however, implemented some sort of communication between human players and bots, this is covered in section \ref{sec:game_communication}

\paragraph{Teammate bots across all genres}
Abraham and McGee created a teammate bot for a simple game: Capture the gunner\cite{abraham10}. The goal of this game is to capture the gunner by touching him from both sides while not being shot. The game required cooperation with their bot because selfish players never passed the first level. Players had great responsibility over the teammate, because of this they found that even if the bot died by the gunner players never felt it was unfair; in fact, some players felt that it was partly their fault if the bot died.

\paragraph{Player Classification}
To beat an enemy in an RTS game, as a team or a single player, you need a strategy that exploits the opponent's weaknesses while eliminating your own (team’s) weaknesses. Both tactics and strategy needs to be good to beat an enemy—although when human beginners play against each other one can win with either just good strategy or good tactics.

To find the teammate’s and opponents’ weaknesses one can make use of simple classification rules, or make it more advanced and use a model. These weaknesses are used to either exploit the opponent's weaknesses or complementing the teammate's weaknesses; these techniques can be used for both purposes, although information gathering and bot action will, however, be different.

\subparagraph{Teammate modeling}
Jansen has created a player model using opponent-modeling for his RTS bot\cite{jansen07}. The bot actions are calculated from a  decision tree computed with supervised learning and neural networks. The goal of the teammate bot is to
\begin{inparaenum}[1\upshape)]
	\item match the number of units and structures the player has, this includes the unit/structure type, e.g. aggressive, defensive;
	\item be able to deduct when the player is under attack from the model;
	\item what action the bot shall take next using the decision tree from the learning strategies;
	\item for the bot to find when the player has either a whole in their defense or attack; and
	\item when the player is switching from defensive to offensive mode, or vice versa.
\end{inparaenum}
He found that his bot could mimic the player; but two problems were identified, the bot could not identify which of the actions were the best one and in addition some players does not always know or do the best action thus the learned actions can use bad strategies.

In their paper\cite{pucheng11} Pucheng and Huiyan uses Q-learning, teammate modeling, and reward allotment for their teammate bot to faster learn which actions leads to a successful goal. Their experiment tested this new learning technique against traditional Q-learning where the bot does not take the teammate into account, and teammate Q-learning where the teammate is taken into account but the reward is not split between bots.

For actually creating a player modeling system, Houlette presents a methodology for how to implement a player model in code\cite{houlette03}. He talks about what a player model is, what it contains, and what it is good at and used for. In addition he gives an simple code example and a description when to update the model and two possible update implementations.

\subparagraph{Opponent modeling}
Kabanza et. al has implemented an RTS bot, HICOR (Hostile Intent, Capability and Opportunity Recognizer). HICOR can, as specified by the name, infer the opponent's intention (i.e. its plan) and use these to analyze the enemy capabilities and opportunities for the bot. Put easy it can infer what build order the enemy is using, what tactic it is using and where it will attack and use this information to its advantage. The underlying system uses Hidden Markov Model to infer the enemy plan.

For recognizing the behavior of the opponent, i.e. aggressive/defensive and what type of aggressive/defensive behavior Schadd et. al uses a hierarchical approach model with two classifiers for their RTS bot. A top level classifier uses fuzzy models to classify the opponent as aggressive vs. defensive, and a bottom level classifier for the type of aggressive/defensive behavior, e.g. the opponent uses mainly tanks or ships when aggressive, or techs when being defensive.

Synnaeve and Bessière RTS bot uses Bayesian networks to model enemy opening strategies\cite{synnaeve11}. The bot learns by watching replays manually label with the strategy of the players.

\section{Teammate Bot in Games}
Teammate bots have been around for quite a while in sports game, such as FIFA\cite{fifa}, but have just started to make a breakthrough in other genres. In most games\cite{callofduty, brotherinarms, rainbow6} the teammate bots cannot be replaced by another player as they either are a part of the story, and thus might not be around all the time, die, or have something else happen to them. In games that are meant be played cooperatively with friends (or strangers), these can be replaced with bots\cite{residentevil5, lostplanet2}.

\subsection{Communication}
\label{sec:game_communication}
Communication has been implemented across several games and genres, most noteworthy are genres where you play as one character, such as FPS games, third-person (TPS) games. In these games some bots communicate you, warning when they spot enemies, get shot, or comes with tips when the player is stuck.

Mass effect, a TPS game, does this in its game by letting the bots tell the player when for example enemies are sighted, and an area is cleared from enemies. Mass Effect 3, goes beyond regular communication and lets players on Xbox 360 to control the bots through voice commands, like a squad leader. This creates a better flow in the game since players do not have to open the action screen (which pauses the game) as often.


\subsection{Controllable bots}
\label{sec:games_controllable}
Today there exists quite a few games that implements the possibility for the player to actively control teammate bots (if the player wants to). We cannot possibly find and go through each game that lets you control its teammate bots, but we will mention a few to show that the feature can be found in games.

Mass Effect\cite{masseffect} does this by having the player the possibility to decide where the bots shall move for cover and hold that position, retreat for cover, even order the usage of certain abilities on target enemies. Rainbow Six Vegas 2\cite{rainbow6} and Brother in Arms: Road to Hill 30\cite{brotherinarms} lets you control its teammate bots much like Mass Effect.


\subsection{RTS Games}
Today, only one RTS game, that we know of, allows to communicate with and control your teammate bot, this game is Red Alert 3\cite{redalert3}. Before describing Red Alert 3, however, a description is given of how teammate bots in RTS games commonly works. These teammate bots acts more or less (depending on the game) on its own, i.e. it does not really collaborate with the player; some bots might try to complement the player's behavior but does not ask if this is the preferred choice for the player. Because commercial games are closed source we do not know to what extent the bot complements the player's behavior, or if they are not taking the player into account at all.

The bot in the first StarCraft\cite{scbw} installment acts entirely on its own, and it does not feel as it behaves differently when playing together with it. In WarCraft 3: The Frozen Throne\cite{wc3ft} the bot reacts to the player coming to aid if s/he is under attack, and communicates its attack position—by pinging on the minimap—to the player when moving out to attack a target. it does this by coming to the players aid if s/he is under attack. Much like WarCraft 3, the bot in StarCraft 2: Wings of Liberty\footnote{First game in the StarCraft 2 trilogy.}\cite{sc2wol} aids the player when s/he is under attack, although it does not ping the minimap when it attacks. In Age of Empires 3\cite{ageofempires3} the bot acts almost entirely on its own; it can, however, request resources from the player and give the player hints.

Red Alert 3\cite{redalert3} on the other hand has the most advanced teammate bot. The game’s campaign mode is played cooperatively with either another human player or bot. The bot can be given simple commands: move to specified position or strike a target; although these have some restrictions as the bot needs to to have the free units to execute the commands. In special missions, the bot will have super weapons that the player will have full control over. Like WarCraft 3 and StarCraft 2, the bot comes to aid the player when it is under attack.

\section{Why StarCraft?}
\label{sec:why_starcraft}	
Why choose StarCraft and not another RTS game? Other games, or engines, the AI could be implemented in is SpringRTS\cite{springrts} which is an RTS game engine, it is by itself not a game and requires a game mod\footnote{A game mod in this case is the set of rules, units, graphics, to create a new RTS game.}, which there are plenty of. ORTS\cite{orts} is aimed for developers and researching, and finally, Wargus\cite{wargus}, a WarCraft II clone that allows for modifications and implementation of an AI. So why not choose one of these instead of StarCraft?

\paragraph{Carefully balanced}
Blizzard Entertainment released StarCraft: Brood War in 1998 and continued to patch it until beginning of 2009\footnote{No official note on this date can be found, the only inofficial page we found that mentioned the date was Wikipedia at: \url{http://en.wikipedia.org/wiki/StarCraft:_Brood_War}, accessed 2012-09-13}. The other games have neither had the time nor the amount of players to carefully balance the game. One factor might be because StarCraft has become a huge E-sport in South Korea\cite{scKotakuKorea}.

\paragraph{Easy to find experienced players}
Because StarCraft have been around for so long and is a commercial successful game it is easy to find experienced players to test the game. By using experienced players as testers, the players do not have to learn the game mechanics and can focus on evaluating the bot instead of the game.

\paragraph{Big community}
StarCraft has a big community, this makes it easy to find and ask people what functionality they would like to see in a teammate bot to gain more ideas, but also evaluate our ideas.

\paragraph{Extending an already existing bot} We have the opportunity to extend an already existing bot, BTHAI\cite{bthai}, for BWAPI\cite{bwapi}. By extending a bot we can focus on making the bot a good teammate and not worry about all the other details; such as good path finding, building placement. Sometimes we, however, improve some already existing systems, e.g. build order, to meet our needs, but we do not have to build the entire system from scratch. In addition, BTHAI is developed by our supervisor, Hagelbäck, and we can therefore get fast help of the system if needed. While we have not searched for other bots to extended, we figured it would be hard to top the support we would get from BTHAI.

\section{Bot strategy}
While our focus lies on communication and conveying intentions, a bot still needs a decent strategy and tactics to win and be useful for the player. There does, however, not exist any specific research on what is a good cooperation strategy that prioritizes the player for RTS games. Instead we will rely on general strategies one player strategies from ``Day[9]''\cite{day9} ([9] is part of the name and not a citation), our own experience playing cooperative games, and by evaluating the bot throughout the development.