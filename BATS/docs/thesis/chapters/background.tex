% !TEX root = ../main.tex
\chapter{Background}
While RTS games has been along since the 80s\cite{adams06, rtsHistory}, only a handful of scientific articles can be found on teammate bots for RTS games, and current big RTS titles have yet to implement a good teammate bot. In general little research has been done for teammate bots throughout all genres; RTS researchers have focused on enemy bots to either create a fun opponent
%: cite!
or to create the best bot to compete in RTS bot tournaments, such as  AIIDE's StarCraft AI Competinion\cite{scaiide} and CIG's StarCraft AI Competition\cite{sccig}.

%: change the paragraph later
In the next section related research topics will be covered, beginning with teammate bots across all genres and asking what guidelines applies to RTS games; continuing with RTS enemy bots and asking, what implementation strategies exists, how a good bot shall play; and ending with communication between AI and humans including design choices to avoid player misinterpretation of the bot. After plowing through all research you will be updated with the current games across all genres using teammate bots and what current RTS games lacks.


\section{Research}
%: Write some text

\subsection{Teammate bots}
As mentioned, little research has been done in the area of teammate bots, especially for RTS games. To our knowledge there exist one survey on teammate bots; in 2010 McGee and Abraham conducted a survey, first presenting their definition of real-time teammate, which their survey is limited to\cite{mcgee10}. Their definition, although summarized, reads; a real-time teammate bot
\begin{enumerate}
	\item works together with team players while taking into account the state, needs, behavior, goals, plans and intentions of these players;
	\item uses coordinated behaviors or decision-making\ldots
	\item {\ldots}that aligns with the team goals;
	\item where these coordinated behaviors or decision-making includes player-related uncertainty requiring inference, reasoning, classification, calculation, or another method; and
	\item whenever possible, prioritizing the player experience.
\end{enumerate}

We will use the same definition in our paper and pointing out if other work is lacking in one or more of these five points.

In their survey\cite{mcgee10} McGee and Abraham noticed that, although human player participation and engagement are one key functionality of a game\cite{reynolds03}, often the player's preferences are neglected and the bot(s) behave what it thinks is the best for either just itself or both the player and itself; this might sound as if it prioritizes the player, but what it does is steer the player into how to play rather than the player steering the bot. When the bot prioritizes the player, some challenges arise; how to create priority rules that do their job correctly\cite{mcgee10}, i.e. the bot has to know, or use a qualified guess, what the player wants. This is no easy task, probably impossible in the near future; humans have a hard time understanding each others intentions, why think AIs that humans have created understands us better\cite{norman07} without even asking?.

An important discovery was the lack of research of communication between human players and bots, “This survey suggests that there are also some aspects of real-time team-mate AI where there seems to be little or no work: ..., and communication.”\cite{mcgee10} Meaning “little or no work” spans through multiple topics and multiple genres; we ourselves have yet to find any topic that talks about communication between human players and bots, for any genre. Some games in other genres has, however, implemented some sort of communication between human players and bots, this is covered below in section \ref{sec:game_communication}

\paragraph{Teammate bots across all genres}
Abraham and McGee created a teammate bot for a simple game: Capture the gunner\cite{abraham10}. The goal of this game is to capture the gunner by touching him from both sides while not being shot. The game required cooperation with their bot because selfish players never passed the first level. Players had great responsibility over the teammate, because of this they found that even if the bot died by the gunner players never felt it was unfair; in fact, some players felt that it was partly their fault if the bot died.


\paragraph{Player Classification}
To beat an enemy in an RTS game, as a team or a single player, you need a strategy that exploits the enemy’s weaknesses while eliminating your own (team’s) weaknesses. Both tactics and strategy needs to be good to beat an enemy—although when human beginners play against each other one can win with just good strategy or good tactics.

To get find the teammate’s and enemies’ weaknesses one can make use of simple classification rules, or make it more advanced and use a model. Articles that focus on complementing the teammate’s weaknesses\cite{jansen07, pucheng11,houlette03} and those that focus on exploiting enemy weaknesses\cite{kabanza10, schadd07, synnaeve11} can be used for both purposes; information gathering and what to execute will, however, be different.

If you strategy is bad, it does not matter how hard you try to win, you will still loose (unless the enemy is even worse). Because you need information, or a model/classification, of a player’s weaknesses, articles that focus on complementing the teammate weaknesses

\section{Teammate Bot in Games}
Teammate bots have been around for quite a while in sports game, such as FIFA\cite{fifa}, but have just started to make a breakthrough in other genres. In most games\cite{callofduty, brotherinarms, rainbow6} the teammate bots cannot be replaced by another player as they either are a part of the story, and thus might not be around all the time, die, or have something else happen to them. In games that are meant be played cooperatively with friends (or strangers), these can be replaced with bots\cite{residentevil5, lostplanet2}.


\subsection{Communication}
\label{sec:games_communication}
Communication has been done in across several games and genres, most noteworthy are genres where you play as one character, such as FPS games, third-person (TPS) games. In these games some bots communicate you, warning when they spot enemies, get shot, or comes with tips when the enemy is stuck.

Mass effect, a TPS game, does this in its game by letting the bots tell the player when enemies are sighted, or area is cleared. Mass Effect 3, goes beyond regular communication and lets players on Xbox 360 to control the bots through voice commands, like a squad leader. This creates a better flow in the game since players do not have to open the action screen (which pauses the game) as often.

\subsection{Controllable}
\label{sec:games_controllable}
Today there exists quite a few games that implements the possibility for the player to actively control teammate bots (if the player wants to). We cannot possibly find and go through each game that lets you control its teammate bots, but we will mention a few to show that the feature can be found in games.

Mass Effect\cite{masseffect} does this by having the player the possibility to decide where the bots shall move for cover and hold that position, retreat for cover, even order the usage of certain abilities on target enemies. Rainbow Six Vegas 2\cite{rainbow6} and Brother in Arms: Road to Hill 30\cite{brotherinarms} lets you control its teammate bots much like Mass Effect.


\subsection{RTS Games}
Today, only one RTS game, that we know of, allows to communicate with and control your teammate bot. First, however, a description is given of how teammate bots in RTS games commonly works. These teammate bots acts more or less (depending on the game) on its own, i.e. it does not really collaborate with the player; some bots might try to complement the player's behavior but does not ask if this is the preferred choice for the player. Because commercial games are closed source we do not know to what extent the bot complements the player's behavior, or if they are not taking the player into account at all.

The bot in the first StarCraft\cite{scbw} installment acts entirely on its own, and it does not feel as it behaves differently when playing together with it. In WarCraft 3: The Frozen Throne\cite{wc3ft} the bot reacts to the player coming to aid if s/he is under attack, and communicates its attack position—by pinging on the minimap—to the player when moving out to attack a target. it does this by coming to the players aid if s/he is under attack. Much like WarCraft 3, the bot in StarCraft 2: Wings of Liberty\footnote{First game in the StarCraft 2 trilogy.}\cite{sc2wol} aids the player when s/he is under attack, although it does not ping the minimap when it attacks. In Age of Empires 3\cite{ageofempires3} the bot acts almost entirely on its own; it can, however, request resources from the player and give the player hints.

Red Alert 3\cite{redalert3} on the other hand has the most advanced teammate bot. The game’s campaign mode is played cooperatively with either another human player or bot. The bot can be given simple commands: move to specified position or strike a target; although these have some restrictions as the bot needs to to have the free units to do execute the commands. In special missions, the bot will have superweapons that the player will have full control over. Like WarCraft 3 and StarCraft 2, the bot comes to aid the player when it is under attack.