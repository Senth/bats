% !TEX root = ../main.tex
% !TEX spellcheck = en_US
\chapter{Conclusions and future work}
First we go through our conclusions on the bot, what the players thought, liked and disliked about
the bot and scenarios. Then we go through some lessons we have learned during the project we thought
was worth mentioning, and last future work for researchers focusing on three different areas,
teammate bot behavior, communication and control.

\section{Bot conclusions}
We conclude that a teammate bot for RTS games that communicates its intentions and reasons is indeed
more fun to play with, although we did not find any statistical difference between Communication +
Control and only Communication both of these include communication.

What surprised us was that some players
preferred when the bot only communicated. This was later identified to depend on the player's skill
where more experienced players liked the ability to control the bot whereas beginners tended to get
overwhelmed by all the existing game action and thus did not want to focus on yet another task:
controlling the bot. This is an interesting question on how to balance the control so that
beginners aren't overwhelmed but at the same time meets the needs of a more skillful player. We talk
more about this in Future work \ref{sec:future_control}.\\


The most liked feature was the ability for the bot to communicate its intentions and reasons to the
player. The most reasons were ``it was fun to know what the bot is doing'' and ``easier to plan
because he told me his plans''. This aligns with what Norman states, humans want to know what the AI
is doing\cite{norman07}.

An important aspect BATS missed was point 5 from McGee's and Abraham's definition\cite{mcgee10} of a
real-time teammate bot, ``whenever possible, prioritizing the player experience''—for the full list
see section \ref{sec:teammate_bots}. When the player either ordered BATS to attack or follow
him/her, BATS almost always ended up blocking the path for the player close to the attack location,
not prioritizing the player experience as it should let the player through.
Disrespecting this ``rule'' probably made ``The army halted in choke points and made it impossible
for me to come through'' top the disliked features list.

\section{Project conclusions}
\paragraph{Magnusson's notes}
Being a perfectionist is not easy, this was one of the causes why we did not meet our first
deadline. I implemented features that were never used but what I thought were nice to have, such as
the Wait Goals, advanced drop mechanics, while we remove Wait goals and drops entirely and it would
not affect the test results.  In addition to this I changed existing systems much more than
needed—instead of using the existing squad manager with squads I create a new squad manager, this
was because the functionality differed quite much what we wanted to have in BATS, but maybe not
needed. We would have saved some weeks on this task alone.

In addition I tweaked and fixed small thing that ``it will only take two hours'', but I found many
of these small tweaks and some of them took more than two hours. After half the bot had been
implemented and the deadline was close, we decided to decrease the number of features and aim for
the next deadline, almost three months later. Now instead for fixing all small bugs and tweaks I
created a ticket in our project manager, then I forced myself to not work on them until all the
bot's main features were implemented. This worked much better and the bot was done, maybe not in no
time, but much faster.

In essence what I learned was that task prioritization is a must if one wants to finish a project,
and then only work on those assigned tasks, if new tasks were found one shall not do them directly
but either put them in the right priority or put them in a bag of things to do if there is time
left.

\section{Future work}
We will split the section in three parts, first discussing future work for the behavior of a
teammate bot, then improvements on the communication, and finally the control of a teammate bot.
While these are RTS game specific topics many of the Communication and Control topics can be
slightly changed to match another game genre.

\subsection{Teammate bot behavior}
BATS behaves in ways that we thought would be good, both through own experience, research, and
testing; but it still lacks a solid behavior to be used in a commercial game as its behavior does
not change depending on the play style of the human player. More research needs to be done on what
players want a teammate bot to do and how much initiative the bot shall have, if the bot has control
the degree of initiative can depend on how much the player want to control the bot, i.e. the more
control the player wants the less initiative the bot shall have, or does the players, independent of
skill and play style, always want the same initiative?

A possibility for teammate bots can be either a full-time or temporarily replacement for a human
player in online matches—temporarily when the player disconnects. But how shall the bot behave when
it is used as a replacement for the entire game? It needs to be balanced so it either matches the
human players' skills in its own team or by setting a predefined skill to either increase or
decrease the entire team's total skill.  When it acts as a temporarily replacement the players might
want it to continue using the same strategy as the player and not do any hasty actions that could
make the player loose, the bot could use some learning algorithm that always is on when the player
plays matches and then send out this information to the other clients if the player gets
disconnected, one of the clients could then take temporary control of the player continue play as
nothing happened.  For both scenarios it might be good to do player modeling for the teammate bot to
know how it shall play, further resources on player modeling for RTS games can be found
here\cite{bakkes09, jansen07, kabanza10, schadd07, synnaeve11}.

\subsection{Communication}
The messages from BATS are very simple and aimed to feel like less auto-generated messages. In case
of text messages the player might not want long ``human-like'' sentences, but instead short
sentences.

There is also the possibility to either synthesize the message or record own messages to see if the
player prefer messages that are read to him/her.  Reading messages out loud might be good but cannot
be processed as quickly by the player and it can be hard if the computer want to send messages very
often, meaning some research needs to be done on how often and what type of messages shall be sent
for the player not to be overwhelmed by messages—this type of research can also be done on text
messages.

\subsection{Control}
\label{sec:future_control}
We concluded that not all commands were used and players preferred control depending on their skill.
One could research which type of commands are preferred and how often they are issued. This could
then serve as another research where more control functionality is either displayed for a player
depending on his/her skill level, this could for example be used in campaigns where players get
introduced to new functionality in the teammate continuously during the campaign, just as s/he is
with new units.

Another experiment could be made how much level of detail the player wants over the commands. E.g.
shall attack make the bot attack at some place (as it does now), shall the player have the ability
to specify where it shall attack, shall the player have the ability to specify which path the attack
shall use, or where the attack shall wait before it reaches its destination for a simultaneous
attack?  This level of degree probably depends on the skill of the player, but as you might have
thought of now this can cause the player to simply play two players, but s/he only needs to control
one bot. This leads to yet another research for a new genre.

Research a new type of RTS genre where the player simply tell the bot what to do, i.e. the player
control the bot entirely but has setup some rules what it should build and how it behaves. This
would be more like a strategy/tactic game where the player neither need good macro or micro, but
good strategy and tactics.

As mentioned further research needs to be done when using the teammate bot as a replacement for
human players in online matches. In regard to control it might be difficult to select who shall
control the bot, or if everyone can control it?
