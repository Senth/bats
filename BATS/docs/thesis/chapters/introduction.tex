% !TEX root = ../main.tex
\chapter{Introduction}
Have you ever wanted to be the boos over your teammate bot\footnote{A bot is a computer player that you either can play with or against} in a game? No!? Then you can stop reading. Today you can find a few new Real-Time Strategy (RTS) games, but none of these lets you control your teammate bot, let alone collaborate with it, at least not to our knowledge; this includes research, where we have not found a single subject on communication between human player and bot.

The focus lies in evaluating if it is worthwhile to do further research in this unexplored area, both for the purpose of researchers and for game developers to help them decide if they shall invest money in a collaborative teammate bot for RTS games.

To familiarize you with the subjects presented a brief description on all subjects are given below. Once you have basic knowledge of the subjects we will give a thorough background what others have done in closely related subjects. Next a detailed description of the methods we used are described; this includes implementation on the bot, why it was implement this way, and how it was evaluated. 
%: Rephrase
The evaluation results are then presented, at the end we conclude the results and presents new subjects of future work.

\section{RTS}
An RTS game is a computer game that heavily relies on strategy and fast executing. An analogy would be to play chess where both you and your opponent can move pieces at the same time, you will only see your own pieces and a tile beside them; meaning you do not know the location of the opponent’s pieces or when s/he makes a move, on top of that you have to think of a strategy fast before your opponent gets a lead.

In reality, RTS games resembles more a battle field than a game of chess—you control the army and it’s unit and on what the resources shall be spent on. From the start of the game you have a single base and your goal is to build up the technology to create powerful units by collecting resources, but you cannot go straight for the best technology because it takes time and if the opponent comes with any attack you will be defenseless against it.

There is no real standard for RTS games, they all work a bit different from each other. Instead of covering all different types of RTS games a short description how StarCraft works will be given to make the reader familiar with the game before talking about more specific problems and solutions.

\subsection{StarCraft}
The AI is implemented in an RTS game: StarCraft, more specifically StarCraft : Brood War—an expansion to StarCraft that introduced new units. Henceforth StarCraft: Brood War will be referenced as StarCraft for simplicity..

%: Import a StarCraft image

\subsubsection{Gameplay}
StarCraft has three races—Terran, Protoss, and Zerg—which the player can choose from. Each of these races have their own unique type of units and their play styles are very different to each other.

\paragraph{Commonalities between races}
All races have worker units that collect minerals and gas—minerals and gas are the games resources used to build structures, units and upgrades. Minerals and gas are usually located in clusters of several mineral fields and maybe one gas. The player always starts at a base with both mineral fields and a gas depot. To mine minerals and gas workers go to the fields and then return to the base building—you can have multiple base buildings. Because of the travel distance you usually want to place your base close to the resource clusters.

Minerals fields have a fixed number of resources they can yield, when all those have been depleted the mineral field disappears. A Vespene gas geyser on the other hand does not disappear after depletion, instead each worker now brings back 2 units of gas each turn, instead of 8. To mine gas an additional structure needs to be placed on the Vespene gas geyser. 

This makes the distance short from the resources to the base which all resources has to be carried to. Only one worker can simultaneously collect minerals from a mineral field, or gas from a gas depot. Minerals have a fixed number of resources to them that can be mined before they disappear; gas on the other hand too have a number of resources, but after that limit is reached gas can still be mined, but now the worker only harvest 2 gas instead of 7. To harvest gas a structure needs to be built on top of the gas depot.

Each unit occupies an amount of supply, ranging from a half supply (Zerglings) to 8 supplies (Carriers). Zerg's Overlord is the only unit that doesn't take any supplies, in fact it gives supplies, this is described in the following text. The maximum supply of units is 200 for all races. To get more supplies than the starting amount, the races have to either build a new base structure or a special supply structure, or unit in the Zerg's case.

\paragraph{Terran}
Terran's worker unit 'SVC' can repair structures, and vehicle units. When they construct a structure they becomes occupied with that task and cannot do anything else. The SVC can however abort the building and let another SVC continue with the construction.

While Protoss's supply structure and Zerg's supply unit has special abilities the Terran's supply unit does nothing more than take up place and give 8 supply to the player.

\paragraph{Protoss}
Protoss units and structures have two health bars oppossed to Terran's and Zerg's one health bar. The first health bar is the base health of the unit, the other is a shield that recharges ones the unit or structure is out of battle. This makes the units and structures strong against small attacks that only damage the units shields, since they can retreat and regain full health again. Note that not all occaissions allow this, since the enemy is usually reluctant to let a protoss army flee to regain full health.

Protoss's worker unit 'Probe' can only mine resources and summon structures, it cannot repair any units as Terran's SVC can. While SVC's are occupied the whole time they construct structures a Probe can simply start to summon a building and it will construct itself. Protoss structures, however, has one restriction, structures needs to be placed under the power of a pylon, pylons are Protoss supply structure and power source. A pylon emits a circle of power and if a structure becomes powerless—the pylon has been destroyed—it will not function and a new pylon has to be built. Pylons, Nexuses (Protoss base), and Assimilator (gas structure) can function without a pylon.

\paragraph{Zerg}

\subsubsection{Balanced}

\section{Teammate}

\subsection{Player}

\subsubsection{General}

\subsubsection{RTS}

\subsection{AI}

\subsubsection{General}

\subsubsection{RTS}

\subsection{Voice Communication}