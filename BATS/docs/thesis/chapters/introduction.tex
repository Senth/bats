% !TEX root = ../main.tex
\chapter{Introduction}
Have you ever wanted to be the boos over your teammate bot\footnote{A bot is a computer player that you either can play with or against} in a game? No!? Then you can stop reading. Today you can find a few new Real-Time Strategy (RTS) games, but none of these lets you control your teammate bot, let alone collaborate with it, at least not to our knowledge; this includes research, where we have not found a single subject on communication between human player and bot.

The focus lies in evaluating if it is worthwhile to do further research in this unexplored area, both for the purpose of researchers and for game developers to help them decide if they shall invest money in a collaborative teammate bot for RTS games.

To familiarize you with the subjects presented a brief description on all subjects are given below. Once you have basic knowledge of the subjects we will give a thorough background what others have done in closely related subjects. Next a detailed description of the methods we used are described; this includes implementation on the bot, why it was implement this way, and how it was evaluated. 
%: Rephrase
The evaluation results are then presented, at the end we conclude the results and presents new subjects of future work.

\section{RTS}
An RTS game is a computer game that heavily relies on strategy and fast executing. An analogy would be to play chess where both you and your opponent can move pieces at the same time, you will only see your own pieces and a tile beside them; meaning you do not know the location of the opponent’s pieces or when s/he makes a move, on top of that you have to think of a strategy fast before your opponent gets a lead.

In reality, RTS games resembles more a battle field than a game of chess—you control the army and it’s unit and on what the resources shall be spent on. From the start of the game you have a single base and your goal is to build up the technology to create powerful units by collecting resources, but you cannot go straight for the best technology because it takes time and if the opponent comes with any attack you will be defenseless against it.

There is no real standard for RTS games, they all work a bit different from each other. Instead of covering all different types of RTS games a short description how StarCraft works will be given to make the reader familiar with the game before talking about more specific problems and solutions.

\subsection{StarCraft}
The AI is implemented in an RTS game: StarCraft, more specifically StarCraft : Brood War—an expansion to StarCraft that introduced new units. Henceforth StarCraft: Brood War will be referenced as StarCraft for simplicity..

%: Import a StarCraft image

\subsubsection{Gameplay}
StarCraft has three races—Terran, Protoss, and Zerg—which the player can choose from. Each of these races have their own unique type of units and their play styles are very different to each other.

\paragraph{Commonalities between races}
All races have worker units that collect minerals and gas—minerals and gas are the games resources used to build structures, units and upgrades. Minerals and gas are usually located in clusters of several mineral fields and sometimes a Vespene geyser which you mine gas from. The player always starts at a base with both mineral fields and a gas depot. To mine minerals and gas workers go to the fields and then return to the closest main structure of the player. Because of the travel distance you usually want to build a new main structure near resource clusters.

Mineral fields have a fixed number of resources they can yield; when all have been depleted the mineral field disappears. A Vespene geyser on the other hand does not disappear after depletion, instead each worker now brings back 2 units of gas each turn, instead of 8. To be able to mine gas in the first place an additional structure needs to be placed on a Vespene geyser. Only one worker can simultaneously collect minerals from a mineral field, or gas from a gas depot, other workers are automatically queued to start mining when the first worker brings back the resources.

StarCraft limits the number of units one can have to 200 supply\footnote{It’s called Psi for Protoss, and Control for Zerg which makes more sense on those unit. Supply is, however, more commonly used jargon for all races.}. Each unit occupies an amount of supply, ranging from a half supply (Zerglings) to 6 supplies (Carriers), armed nuclear silos take up 8 supplies. In general, the better the unit the more supply it takes. The player does, however, only start with 9 to 10 max supply depending on the race. To increase the max supply a special structure (unit for Zerg) can be built; main bases also generate supply, 10 for Terran Command Center, 9 for Protoss Nexus, and 1 for Zerg Hatchery.

\paragraph{Constructing Structures}
Terran's worker unit, SCV,  When an SCV constructs a structure it becomes occupied and cannot do anything else; it can still be targeted, and can stop building and let another SCV take over—this can be the case when an enemy attacks the SCV.

Zerg’s worker unit, Drone, will instead morph itself into a structure. The drone will disappear once the morphing stage begins, but if the player cancels the building progress the drone will reappear. Once the morph completes the drone is forever lost. Zerg structures need to be placed on creep which Hatcheries and colonies help to spread. The only exception to this is Hatcheries and Extractor (for Vespene geyser).

Protoss’s worker unit, Probe, need to construct structures in a psionic power grid that is generated by pylons structures. If a structure becomes unpowered it will not function again until a nearby pylon is reconstructed to create a new psionic power grid.

\paragraph{Constructing Units}
Terran and Protoss function in basically the same way: they both have separate unit producing structures for different kinds of units, e.g. Barracks/Gateway for basic ground units, Starport/Stargate for flying units, and so forth. All units, however, cannot be built from these structures directly, some units require another stand-alone structure, like Acadamy for Firebats and Medics. The main structure (Command Center and Nexus) can only build workers. Only one unit can be built simultaneously from these structures, if one wants a huge production of units, massive amount of unit producing structures are needed.

Zerg on the other hand only has one unit producing structure, the Hatchery, that pops out larvae in a certain time interval—one Hatchery can maximum have three larvae. These larvae are then morphed into the unit of choice. To build other units than Drones and Overlords one must build the structure needed for the unit, e.g. building a Spawning Pool to be able to build Zerglings from a Hatchery, in a similar way Acadamy is needed for building Firebats from a Barracks.

While it seems most obvious to build “the best” units at the beginning, this is not possible. The “best” units, although there is no such thing as best as all units are good depending on the situation, require certain structures, that in turn require other structures. For example, to build a Carrier one requires a Fleet Becon (for the Carrier tech) and a Stargate (to build the unit), the Stargate then requires a Cybernetics core, the Cybernetics core requires a Gateway and a Gateway requires a Nexus (although one has a Nexus from the start). As you can see you need to put quite a lot of resources just to be able to build a Carrier, if you go straight for Carriers your enemy can just waltz into the base with any number of units and kill the entire base.

\paragraph{Unit Health}

%: Notes below

\paragraph{Protoss}
Protoss units and structures have two health bars oppossed to Terran's and Zerg's one health bar. The first health bar is the base health of the unit, the other is a shield that recharges ones the unit or structure is out of battle. This makes the units and structures strong against small attacks that only damage the units shields, since they can retreat and regain full health again. Note that not all occaissions allow this, since the enemy is usually reluctant to let a protoss army flee to regain full health.

Protoss's worker unit 'Probe' can only mine resources and summon structures, it cannot repair any units as Terran's SVC can. While SVC's are occupied the whole time they construct structures a Probe can simply start to summon a building and it will construct itself. Protoss structures, however, has one restriction, structures needs to be placed under the power of a pylon, pylons are Protoss supply structure and power source. A pylon emits a circle of power and if a structure becomes powerless—the pylon has been destroyed—it will not function and a new pylon has to be built. Pylons, Nexuses (Protoss base), and Assimilator (gas structure) can function without a pylon.

\paragraph{Zerg}

\subsubsection{Balanced}

\section{Teammate}

\subsection{Player}

\subsubsection{General}

\subsubsection{RTS}

\subsection{AI}

\subsubsection{General}

\subsubsection{RTS}

\subsection{Voice Communication}