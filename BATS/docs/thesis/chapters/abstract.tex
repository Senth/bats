% !TEX root = ../main.tex
% !TEX spellcheck = en_US
\abstract
\begin{changemargin}{+2cm}{+2cm}
\noindent
Communication in team games between human player is important, but has been disregarded in teammate
bots for Real-Time Strategy (RTS) games. Control over the teammate bots have existed for a while in
other genres than RTS and if implemented correctly it adds another dimension to the game.

In this
study we investigate whether players think it is more fun to play with a bot that communicates its
intentions and reasons and is controllable by the human player. But also what features are liked,
disliked, missed to provide guidelines to future researchers and companies.

For this we create a StarCraft RTS bot with communication and control abilities. The experiment
consists of four scenarios, turning on/off communication and control, to conclude what players think
is fun.

All testers agreed on communication being important and more fun to play with. Beginners did,
however, not like the control feature as they already had enough on their mind whereas experienced players
preferred having some control over the bot. 

We conclude that communication is an important role in team games, including RTS games. More work
needs to be done how to integrate control so that beginners do not feel overwhelmed and at the same
time experienced players do not feel restrained by too simple control commands.

\par\vspace {1cm}
% 3-4 keywords, maximum 2 of these from the title, starts 1 line below the
% abstract.
\noindent
\textbf{Keywords:} AI, RTS, StarCraft, communication.
\end{changemargin}
\clearpage
