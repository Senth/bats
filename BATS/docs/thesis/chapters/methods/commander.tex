% !TEX root = ../../main.tex
% !TEX spellcheck = en_US
\section{Commander}
The Commander creates all orders for BATS, except defenses as the \nameref{sec:defense_manager} does that. This means that it orders attacks, drops, expansions, scouts, and when to transition to the next phase.

\subsection{Orders/Commands}
Below all orders are described and how they work depending on the current state of BATS. In addition some orders can have a slightly different behavior depending if it was BATS or the teammate who ordered it. Because of all tests and different behaviors in the command these are described with mostly pseudo-code. Common for all order functions are the two parameters: \texttt{alliedOrdered} and \texttt{reason}; alliedOrdered is set to true if the teammate ordered the command, reason is the reason to print out if the command is successful, this is only used when BATS ordered the command—e.g. expand because BATS expands.

\paragraph{Attack}
Has in general three behaviors: Creates a new frontal attack, reinforces the existing frontal attack, or does nothing.
\begin{lstlisting}[label={lst:order_attack},caption={Pseudo-code of the attack command}]
// Never do a frontal attack when under attack
if (isUnderAttack()) {
	if (alliedOrdered) {
		mIntentionWriter->write(BotAttackNot, BotIsUnderAttack);
	}
	return;
}

// Teammates can create attacks even if few units
if (alliedOrdered) {
	canAttack = !freeUnits.empty();
} else {
	canAttack = canFrontalAttack(); // Checks enough units
}

if (canAttack) {
	oldSquad = mSquadManager->getFrontalAttack();
	
	// Add free units to the old attack squad if it exists
	if (oldSquad != NULL) {
		oldSquad->addUnts(freeUnits);
		mIntentionWriter->write(BotAttackMerged, reason);
	}
	// Create new attack and ping position
	else {
		attackSquad = newAttackSquad(freeUnits);
		attackPos = attackSquad->getAttackPosition();
		mIntentionWriter->write(BotAttack, reason, attackPos);
	}
} else {
	mIntentionWriter->write(BotAttackNot, BotNotEnoughUnits);
}
\end{lstlisting}

\paragraph{Follow}

\paragraph{Drop}

\paragraph{Scout}

\paragraph{Expand}

\paragraph{Transition}

\subsection{Order creation rules}
The Commander can create orders either from its own actions and states, or what the allied player is doing. Examples of this is it might attack when it is expanding, and it might expand if it has high amount of minerals, for reaction to allied actions it attack if the allied player is expanding.

\subsubsection{Reacting on own actions and states}
The commands are ordered by commands in own reactions.

\paragraph{Expand}
In order to expand from own reactions BATS meet all conditions below.
\begin{enumerate}
	\item Not be under attack—expanding when we are under attack will most likely kill the expansion directly
	\item Not already be expanding—we do not want to spam expansions
	\item Active expansion count needs to be less than \commanderExpansionActiveMax—good number of active bases as it keeps a large and steady income to support many unit producing structures. Active expansions are expansions where at least \classificationExpansionExpansionMineralsLow~minerals left.
	\item No new expansion should have been build in \commanderExpansionIntervalMin, again we do not want to spam expansions and this seemed like a reasonable time, but has not been extensively tested.
\end{enumerate}
When all four conditions are met it will check if any of the following conditions are met, if they are an expansion will be added to the beginning of the build order.
\begin{enumerate}
	\item BATS has an attack—good idea to expand when attacking (or vice versa), distracts the enemy from the expansion long enough for the expansion to complete\cite{day9}.
	\item Expansions are saturated—generic expansion rule to expand when low on expansions, i.e. below max active expansion count. Expansions are saturated tests if the number of workers per mineral patch (thus per expansion) is at or above \classificationExpansionWorkersPerMineralSaturation !!! CITE !!!\marginpar{CITE}. % Cite workers per mineral saturation from team liquid
	\item An expansion is running low on minerals—making sure we stay on the same number of active expansions. This is the same as an expansion that is not active (i.e. having less than \classificationExpansionExpansionMineralsLow~minerals left) but still having some minerals left to be mined.
	\item High on minerals—if we have too much minerals that means we cannot build enough structures or units, thus we can as well add another base and possibly increasing the amount of gas mined. Activates when the mineral count is above \classificationHighOnMinerals.
\end{enumerate}

\paragraph{Attack}
To attack from own reactions BATS shall meet all conditions below.
\begin{enumerate}
	\item Shall not be under attack—it is better to stay and defend and then possibly attack, although a small counter attack might be effective when the enemy army is small and BATS can spare some units for the attack, this functionality has not been implemented.
	\item Not have a current attack—having one big army is easier to control and generally much stronger than splitting it into 2 smaller attacks for slower armies\cite{day9} as BATS will use in the experiment. Although ordering a second attack when we have a frontal attack will reinforce units, but we did not have enough time to implement and test when to reinforce the army.
\end{enumerate}
When all two conditions are met it will check if any of the following conditions are met, if they are an attack will be created.
\begin{enumerate}
	\item Expanding—as explained earlier under the expand, it is good to attack while expanding.
	\item Upgrade soon done—when an upgrade will finish soon this means that the bot will get stronger and it might be possible that the enemy has either not caught up in upgrades, i.e. the longer we wait the less valuable the upgrade is\cite{day9}. Upgrade soon done checks if any of the free units that could be used for an attack will be affected with an upgrade that is currently upgrading and if the upgrade finishes within \classificationUpgradeSoonDone.
\end{enumerate}

\paragraph{Scout}
The scout order conditions are much simpler than expand and attack. BATS will always send out a scout if it it is not scouting, not under attack and has \commanderScoutOnWorkerCount.

\paragraph{Transition}
The transition order conditions are just as simple as the scout. IT will transition if it is not already in late game, no buildings in the build order, and is high on resources. It is high on resources when both the mineral and gas count is higher or equal to \classificationHighOnMinerals~and \classificationHighOnGas~respectively.

\subsubsection{Reacting on allied actions}
All orders here are grouped by allied actions instead of the commands (as reacting on own actions were), this feels like a more intuitive approach and was implemented this way.

\paragraph{Allied expands}
If the allied player expands, BATS first want to create a distracting attack (for now only drops are available) to distract the enemy from the expansion, if no drops are available it will create a frontal attack. But before the commander creates any attack it checks so that we are not under attack or already attacking.

\paragraph{Allied attacks}
Depending on what type of attack the allied player has BATS behaves differently, but common for all types of attack, it will not do anything if it is under attack. If the allied attacks with a frontal attack BATS will first try to join it, if it already does not have a frontal attack and has enough units (\classificationFrontalAttackUnitsMin) for a frontal attack, otherwise it will try to drop if it does not have a drop out and has enough units for a drop.

When the allied player has a distracting attack out BATS will too try to create a distracting (drop) attack, but will only succeed if it has enough units.
