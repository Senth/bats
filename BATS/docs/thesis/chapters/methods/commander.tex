% !TEX root = ../../main.tex
% !TEX spellcheck = en_US
\section{Commander}
\label{sec:commander}
The Commander creates all orders for BATS, except defenses as the \nameref{sec:defense_manager} does that. This means that it orders attacks, drops, expansions, scouts, and when to transition to the next phase.

\subsection{Orders/Commands}
Below all orders are described and how they work depending on the current state of BATS. In addition some orders can have a slightly different behavior depending if it was BATS or the teammate who ordered it. Because of all tests and different behaviors in the command these are described with mostly pseudo-code. Common for all order functions are the two parameters: \texttt{alliedOrdered} and \texttt{reason}; alliedOrdered is set to true if the teammate ordered the command, reason is the reason to print out if the command is successful, this is only used when BATS ordered the command—e.g. attack because BATS expands. In additions all commands write out the intention (and possibly reason) of the command, this is done through IntentionWriter \texttt{mIntentionWriter}; how IntentionWriter works is described in section \ref{sec:communication}

\paragraph{Attack}
Has in general three behaviors: Creates a new frontal attack, reinforces the existing frontal attack, or does nothing. Listing \ref{lst:order_attack} displays the pseudo-code for the attack command.
\begin{lstlisting}[label={lst:order_attack},caption={Pseudo-code for the attack command}]
// Never do a frontal attack when under attack
if (isUnderAttack()) {
	if (alliedOrdered) {
		mIntentionWriter.write(BotAttackNot, BotIsUnderAttack);
	}
	return;
}

// Teammates can create attacks even if few units
if (alliedOrdered) {
	canAttack = !freeUnits.empty();
} else {
	canAttack = canFrontalAttack(); // Checks enough units
}

if (canAttack) {
	oldSquad = mSquadManager.getFrontalAttack();
	
	// Add free units to the old attack squad if it exists
	if (oldSquad != NULL) {
		oldSquad.addUnts(freeUnits);
		mIntentionWriter.write(BotAttackMerged, reason);
	}
	// Create new attack and ping position
	else {
		attackSquad = new AttackSquad(freeUnits);
		attackPos = attackSquad.getAttackPosition();
		mIntentionWriter.write(BotAttack, reason, attackPos);
	}
} else {
	mIntentionWriter.write(BotAttackNot, BotNotEnoughUnits);
}
\end{lstlisting}

\paragraph{Follow}
Follow almost behaves as the attack command, but instead of creating a regular attack squad, it will instead follow the largest teammate player’s squad. In addition to the attack’s three behaviors—create new frontal attack (that follows teammate), reinforce the frontal attack, or does nothing—it has one additional behavior: If BATS already has a frontal attack, that is not following a teammate, it will abandon its goal and instead follow the teammate.

\paragraph{Drop}
Drop will try to create a drop from any of the unit compositions that are available.
\begin{lstlisting}[label={lst:order_drop},caption={Pseudo-code for the drop command}]
availableCompositions = getUnitCompositionByType(freeUnits, Drop);

// Create drop
if (!availableCompositions.empty()) {
	randomId = rand() % availableCompositions.size();
	chosenComposition = availableCompositions[randomId];

	dropSquad = new DropSquad(freeUnits, chosenComposition);
	mIntentionWriter.write(BotDrop, reason, dropSquad.getDropPosition());
}
// No drops available
else {
	if (alliedOrdered) {
		mIntentionWriter.write(BotDropNot, BotNotEnoughUnits);
	}
}
\end{lstlisting}

\paragraph{Scout}
As with drop, scout uses available unit composition; it does however not choose a random composition but uses the composition with highest priority—i.e. the first composition, as UnitCompositionFactory’s getUnitCompositionByType() returns composition sorted by priority, where the highest priority is first.

\paragraph{Expand}
Expand will first check if there are any expansions available, i.e. not already taken; if there are expansions available it will append a command center (or equivalent for other races) to the build planner, which in turn will build an expansions when it has resources for it.

\paragraph{Transition}
Transition transitions to the next phase in the game—there are three phases: early, mid, and late
game. How the commander handles the transition command can be seen in listing
\ref{lst:order_transition}.

\subsection{Order creation rules}
The Commander can create orders either from its own actions and states, or what the allied player is doing. Examples of this is it might attack when it is expanding, and it might expand if it has high amount of minerals, for reaction to allied actions it attack if the allied player is expanding.
\begin{lstlisting}[label={lst:order_transition},caption={Pseudo-code for transition command}]
// Only transition if not in late game already
if (mBuildPlanner.canTransition()) {
	if (mBuildPlanner.getCurrentPhase() == "early") {
		intention = BotTransitionMid;
	else if (mBuildPlanner.getCurrentPhase() == "mid") {
		intention = BotTransitionLate;
	}

	mBuildPlanner.switchToNextPhase();
} else {
	if (alliedOrdered) {
		mIntentionWriter.write(BotTransitionNot, BotTransitionNoMore);
	}
}
\end{lstlisting}

\subsubsection{Reacting on own actions and states}
These are the reactions (commands) to BATS's own state.

\paragraph{Expand}
In order to expand from own reactions BATS meet all conditions below.
\begin{itemize}
  \item \textbf{Not be under attack.} Expanding when we are under attack will most likely kill the expansion
	  directly
	\item \textbf{Not already be expanding.} We do not want to spam expansions
	\item \textbf{Less than \commanderExpansionActiveMax~active expansions.} Good number of
	  active bases as it keeps a large and steady income to support many unit producing structures.
	  Active expansions are expansions where at least
	  \classificationExpansionExpansionMineralsLow~minerals left.
	\item \textbf{No new expansion shall have been build in \commanderExpansionIntervalMin.} Again we do not
	  want to spam expansions and this seemed like a reasonable time, but has not been extensively
	  tested.
\end{itemize}
When all four conditions are met it will check if any of the following conditions are met, if they
are an expansion will be added to the beginning of the build order.
\begin{itemize}
	\item \textbf{BATS is attacking.} It is good to expand when attacking (or vice versa) as this
	  distracts the enemy from the expansion long enough for the expansion to complete\cite{day9}
	\item \textbf{Expansions are saturated.} This tests if the number of workers per mineral patch is at or
	  above \classificationExpansionWorkersPerMineralSaturation. Simply put, even if you add more
	  workers to mine, they will have to wait longer for the mineral patch to be free, thus it will
	  not increase your income. This rule will allow BATS to expand early in the game when it has
	  less than \commanderExpansionActiveMax~expansions.
	\item \textbf{An expansion is running low on minerals.} Making sure we stay on the same number of active
	  expansions. This is the same as an expansion that is not active (i.e. having less than
	  \classificationExpansionExpansionMineralsLow~minerals left) but still having some minerals
	  left to be mined.
	\item \textbf{High on minerals.} If we have too much minerals that means we cannot build enough structures
	  or units, thus we can as well add another base and possibly increasing the amount of gas
	  mined. Activates when the mineral count is above \classificationHighOnMinerals.
\end{itemize}

\paragraph{Attack}
To attack from own reactions BATS shall meet all conditions below.
\begin{itemize}
  \item \textbf{Shall not be under attack.} It is better to stay and defend and then possibly
	attack, although a small counter attack might be effective when the enemy army is small and BATS
	can spare some units for the attack, this functionality has not been implemented.
  \item \textbf{Not have a current attack.} aving one big army is easier to control and generally
	much stronger than splitting it into 2 smaller attacks for slower armies\cite{day9} as BATS will
	use in the experiment. Although ordering a second attack when we have a frontal attack will
	reinforce units, but we did not have enough time to implement and test when to reinforce the
	army.
\end{itemize}
When all two conditions are met it will check if any of the following conditions are met, if they
are an attack will be created.
\begin{itemize}
	\item \textbf{Expanding.} As explained earlier under the expand, it is good to attack while
		expanding.
	\item \textbf{Upgrade soon done.} When an upgrade will finish soon this means that the bot will
		get stronger and it might be possible that the enemy has either not caught up in upgrades, i.e.
		the longer we wait the less valuable the upgrade is\cite{day9}. Upgrade soon done checks if any
		of the free units that could be used for an attack will be affected with an upgrade that is
		currently upgrading and if the upgrade finishes within \classificationUpgradeSoonDone.
\end{itemize}

\paragraph{Scout}
The scout order conditions are much simpler than expand and attack. BATS will always send out a
scout if it is not scouting, not under attack, and has \commanderScoutOnWorkerCount. The commander
uses \ref{sec:unit_composition}s for the scout to get the best available scout unit. From lowest
priority this is:
\vspace{0.5em}
\begin{compactenum}
  \item SCV
	\item Marine
	\item Vulture
	\item Wraith
\end{compactenum}
Wraiths are flying units with cloak ability (if researched) this make them an ideal scout as they
fly over terrain, are quite fast, and can cloak if they get too close to the enemy. Vultures are on
the other hand the fastest ground unit for Terran. See section \ref{sec:scout_squad} how squads
work, listing \ref{lst:unit_comp_scout} for the unit composition config file, and
\ref{sec:unit_composition} for how unit compositions work.

\paragraph{Transition}
The transition order conditions are just as simple as the scout. It will transition if it is not
already in late game, no buildings in the build order, and is high on resources (higher or equal to
\classificationHighOnMinerals~minerals and \classificationHighOnGas~gas).

\subsubsection{Reacting on allied actions} All orders here are grouped by allied actions instead of
the commands (as opposed to reacting on own actions), this feels like a more intuitive approach and
was implemented this way.

\paragraph{Allied expands} If the allied player expands, BATS first want to create a distracting
attack (for now only drops are available) to distract the enemy from the expansion, if no drops are
available it will create a frontal attack. But before the commander creates any attack it checks so
that we are not under attack or already attacking.

\paragraph{Allied attacks} Depending on what type of attack the allied player has BATS behaves
differently, but common for all types of attack, it will not do anything if it is under attack. If
the allied attacks with a frontal attack BATS will first try to join it, if it already does not have
a frontal attack and has enough units (\classificationFrontalAttackUnitsMin) for a frontal attack,
otherwise it will try to drop if it does not have a drop out and has enough units for a drop.

When the allied player has a distracting attack out BATS will try to create a distracting (drop)
attack as well, but will only succeed if it has enough units.
