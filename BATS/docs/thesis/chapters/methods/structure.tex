% !TEX root = ../../main.tex
% !TEX spellcheck = en_US
We begin by describing the main features, behaviors, and goals of our bot; these features have been
selected by either what players wanted, what we think would be a good feature, or both and then
evaluated. We then give a brief overview of the system and how the different components/classes work
and communicate with each other to get a quick grasp how everything is connected.

After the overview we will give a detailed description of all the important features of the classes,
giving motivations of our design choices. The design choices were selected by either research,
common knowledge in the StarCraft community, what we think would be good, or a mix of any and then
evaluated, first on ourselves and then, close to the end, on our test player to find weird and
non-acceptable behaviors.

\paragraph{Configurable variables}
Fixed variables in BATS (BTH (Blekinge Tekniska Högskola) AI Teammate StarCraft), our proposed bot
for this thesis, are configurable from a config.ini file. This design choice was made to easily
tweak these variables in-game, instead of having to recompile, start the game, wait until the game
state is present (necessary buildings have been built) and then see the results. The configuration
file could be reloaded anytime in-game using \emph{/reload config} command. Throughout this chapter
these will be presented with their current value, variables that are configurable will have a
superscript of \conf attached to it, e.g. 150\conf~seconds. If a range has a superscript, e.g. [0.0,
1.0]\conf, both 0.0 and 1.0 is configurable.

\paragraph{Terminology explained}
There are a couple of words that needs to be explained before digging into the details.

\paragraph{Region \& choke point} A typical map from StarCraft contains many regions, the regions
are (often) big open spaces divided by choke points—narrow paths, e.g. paths moving up or down a
cliff. A region could be fully isolated, this means it can only be reached by air—BATS does not work
on these kind of maps, all regions must be connected in one way or the other for BATS to work
properly. Regions and choke points are illustrated in figure \ref{fig:region_and_choke_points}—this
is also the map used for all experiments. Regions and choke points are automatically generated by
BWTA\cite{bwta} in BWAPI\cite{bwapi}.
\paragraph{Squad} Units grouped together with a common goal or behavior, such as Attack squad or Scout squad.
\paragraph{Teammate/enemy squad} A virtual squad used to group close allied or enemy units together.
BATS uses this information to deduct certain information from these players.
\paragraph{Expanding} The time from when one wants to build another base (to mine minerals and gas
from) until the base has completed.

\begin{figure}[htb]
	\centering
	\includegraphics[width=0.8\textwidth]{space_atoll_(co-op)_regions_small}
	\caption[Regions and choke points]{
		Regions and choke points.
		\usebox{\LegendLineLightGreen} Region borders;
		\usebox{\LegendLineRed} choke points; and
		\usebox{\LegendBoxYellow} unwalkable terrain.}
	\label{fig:region_and_choke_points}
\end{figure}
