% !TEX root = ../../main.tex
% !TEX spellcheck = en_US
\section{Exploration manager}
\label{sec:exploration_manager}
The exploration manager in BATS uses most of the functionality from BTHAI, but some functionality has been changed, added, and removed to work better with BATS. The main function of the exploration manager tracks when a region and expansion was last checked. It does this by by checking if either a region’s center or an expansion position is visible, if it is, it will automatically update the time of the last visit. This information is used by both \nameref{sec:scout_squad} to get the next scout location—i.e. the one with oldest visit time—and \nameref{sec:attack_coordinator} to get expansions that have not been visited for \attackCoordinatorExpansionNotCheckedTime—used for attacks when no structures have been found. How \nameref{sec:scout_squad} and \nameref{sec:attack_coordinator} works are described in section \ref{sec:scout_squad} and \ref{sec:attack_coordinator} respectively.

Exploration manager saves all structures it spots, because BATS does not have full map vision and thus needs to save the information of the enemy. This information is used by \nameref{sec:attack_coordinator} to decide a place to attack and \nameref{sec:attack_squad} to find a common spot to attack when it is close to the target location.

\subsection{Resource counter}
\label{sec:resource_counter}
Although the Exploration Manager handles where to explore and enemy structures placements, it does not handle how many resources there are in each base. This is done by the resource counter, although it currently only counts the number of minerals. In the start of the game it goes through all static mineral locations and set their mineral count to their initial value (1500 usually). Whenever a mineral field is visible it will update the value of that mineral field. It will, however, not try to calculate how many minerals there might be currently if the mineral field is not visible (as it could by checking the rate of the decline).

To keep track of which resource belong where the resource counter has resource groups, which in turn contains the mineral fields. Both the Resource and ResourceGroup classes have the ability to return current and initial number of resources, in addition it can calculate how much of the resources are left in fractions in the interval of [0.0, 1.0].