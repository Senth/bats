% !TEX root = ../../main.tex
\section{Defense Manager}
\label{sec:defense_manager}
The defense manager handles the defense of both defending its own base and the player’s base. 



\paragraph{Defense locations}
The defense locations are calculated from choke points that abut either BATS’s or the allied’s currently occupied regions. It does not defend choke points to empty regions where this region only abuts to BATS or allied occupied region, this can be seen in figure \ref{fig:defense_locations} where the empty region is outlined in green.

The defense locations are defended by \nameref{sec:hold_squad}s and a \nameref{sec:patrol_squad}. The hold squads, described fully in section \ref{sec:hold_squad}, are stationary and defend the choke point from a certain roaming perimeter, where the center is calculated by finding a position in the range \squadDefendRoamDistanceMinMax~closest to one of BATS’s or the allied’s buildings and the radius being \squadDefendRoamPerimeter. Units from the hold squad will stay in the roaming perimeter until enemy units enter the defense perimeter (\squadDefendDefendPerimeter~in radius), they will then start to attack the enemy. 

Patrol squads on the other hand contains all free units, these will patrol between the defense locations. If enemy units enter the enemy offensive perimeter (\squadDefendEnemyOffensivePerimeter~in radius) in a defense location all squad units will move to the defend location to defend it. If the defense manager thinks the enemy is too strong (the enemy has more units in number of supplies) it will disband the other hold squads making the units in these squads join with the patrol squad.

Both hold squads and patrol squads roam or patrol between BATS’s defense location so the bot won’t disturb the allied player with its units. But it will still come and defend if the enemy attacks the allied player, if BATS is not under attack itself. In addition to defending the defend locations, the defense manager will send out the patrol squad to defend locations inside BATS’s or the allied’s base that are under attack, for example if the enemy drops inside the base. This will, however, never bring the hold squads as they should defend the entrance to the base.

Figure \ref{fig:defense_locations} displays an example of where Defend locations can be, how regions are interpreted, and so forth.

%: Add picture with locations that are defended, including roam positions
\begin{figure}[htb]
\centering
\caption{Example of defense locations.}
\label{fig:defense_locations}
\end{figure}

Green outlined regions are BATS’s regions, blue outlines are allied’s regions, regions outlined with both green and blue belong to both. Yellow regions are non-defended as they abut only BATS or allied regions. Defense locations are outlined with