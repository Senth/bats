% !TEX root = ../../main.tex
\section{Build planner}
\paragraph{Purpose}
Build planner is the core for planning structure and army creations. Structure includes constructing structures, upgrades, add-ons and army refers to all combat and non-combat (e.g. Medic) units. The production process during the whole game is divided into three phases namely early, mid, and late. BATS follows specific build order provided in the configuration file (/BATS-data/buildorder/) and can be changed by the player, but was not during the experiments. The idea is that the player shall freely be able to change early, mid, and late game build orders to match his play during these phases, this can be done both before or within the match (but not in the experiments as stated, as we thought this would be too much to think of).

\paragraph{Commands used}
The phase transition during the game is done by the command transition or transition BUILD\_NAME The configuration in the transition file is overridden by the command transition BUILD\_NAME.

\paragraph{How it works}
The priority of the production is the order in which the units are placed in the configuration file. The army is classified into \emph{must have} and \emph{percentage} units for each phase; must have units are always prioritized and in the order they were entered, when no more must have units need to be build it will instead try to maintain the number of percentage units.

A short example of the format is given below in listing \ref{lst:build_order_example}.

\begin{lstlisting}[label={lst:build_order_example},caption={Build order example file},language=ini]
<name>
Marine Medics
<description>
Builds marine and medics for a fast offense.
<build-order>
Terran_Supply_Depot
Terran_Barracks
Terran_Refinery
Terran_Academy
Terran_Factory
<units>
80\% Terran_Marine
20\% Terran_Medics
<must-have>
2 Terran_Medics
1 Terran_Vulture ; for scouting
\end{lstlisting}

\begin{function_description}
	\item[\textless name\textgreater] Name of the build order and its strategy.
	\item[\textless description\textgreater] Description of the strategy.
	\item[\textless build-order\textgreater] List of structures to be built during this phase, the structures are built in the order they are listed.
	\item[\textless units\textgreater] Defines the army the current phase shall have and is specified in percentage. It will always prioritize the unit that is furthest away from its goal. For example using the values from the example build order, the current army consists of 74\% marines and 18\% medics. This will prioritize the medics, why? Marines are $74/80 =  0.925$ (7.5\%) from their goal whereas medics are $18/20 = 0.9$ (10\%) from their goal.
	\item[\textless must-have\textgreater] Compulsory units that are to be built during a phase.
\end{function_description}

If a unit with the highest priority cannot be built because of
\begin{inparaenum}[a\upshape)]
	\item low resources, wait for the resources to build and and do not build another unit; or
	\item no structures are free to build the unit, build the next unit in the priority list instead.
\end{inparaenum}

Destroyed structures and units are automatically readded to the build order, meaning they would be built again.
