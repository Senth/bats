% !TEX root = ../../main.tex
\section{Build planner}
\subsubsection{Purpose}
Build planner is the core for planning Structure (Building) and Army creations. Here Building or Structure includes production, upgrade, add-on and resource gathering buildings and Army refers to all combat and non-combat (Medic) type units. The production process during the whole game is divided into three phases namely early, mid and late. BATS follows specific build order provided in the configuration file (/BATS-data/buildorder/) by the user. User can configure and keep more than one build order for a phase. The exact transition graph is configured using transition config file and the individual phase build order is configured using files in early, mid and late folders.

\subsubsection{Commands used}
The phase transition during the game is done by the command transition or transition FILE\_NAME. The configuration in the transition file is overriden by the command transition FILE\_NAME. BATS asks user for the permission to make transition in case if the user forget to do so.

\subsubsection{How it works}
The priority of the production is the order in which the units are placed in the configuration file. The Army is classified into Must Have and Percentage units for each phase. The priority of army production is done based on maintenance of percentage and higher priority is given to the Must Have units over Percentage units.

\subsubsection{Example and Format}
{\it
\textless name\textgreater\newline
early-x\newline
\textless available-transitions\textgreater\newline
mid-x\newline
mid-Y\newline
\textless build-order\textgreater\newline
Terran\_Supply\_Depot\newline
Terran\_Barracks\newline
Terran\_Refinery\newline
Terran\_Bunker\newline
Terran\_Academy\newline
Terran\_Factory\newline
\textless units\textgreater\newline
80\% Terran\_Marine\newline
20\% Terran\_Medic\newline
\textless must-have\textgreater\newline
2 Terran\_Medic\newline
1 Terran\_Vulture\newline}

Above is the default configuration followed by the BATS for phase "early-x" which can be edited by the user. Separate configuration is available for separate phase namely "early", "mid" and "late". The configuration file has three parts name, available-transition, build-order, units and must-have. 
\paragraph{\textless name\textgreater} has the phase name to which the configuration file applies to. 
\paragraph{\textless available-transition\textgreater} has the possible phases to which the current phase can transit to. 
\paragraph{\textless build-order\textgreater} has the list of structure or building names to be built for that phase. The structures are built in the order they are listed. 
\paragraph{\textless units\textgreater} defines the Army that the current phase should have and is mentioned in percentage.
It will always prioritize the unit that is furthest away from its goal.\newline\newline
E.g. 60\% marines are at (56\%), 20\% tanks are at (18\%) will choose to prioritize a tank, why?\newline
Marines: 56/60 = 0.933... meaning roughly 7\% from its goal.\newline
Tanks: 18/20 = 0.9 meaning 10\% from its goal.\newline
Tanks are then furthest away from its goal.\newline
\textless must-have\textgreater lists the units that are compulsory to be available during a phase. Units are listed in decimal numbers. If the highest priority unit cannot be built because of:\newline
- Low minerals -\textgreater Wait for minerals, do not build another unit\newline
- No buildings are free (or does not exist) -\textgreater Build next unit in the priority list instead.\newline
BATS maintains the production by considering the unit destruction and building availability to produce that unit. i.e. when a must have unit is destroyed, it is rebuilt again and when a percentage unit is destroyed, it is added again into the production queue.\newline
BATS has pre-defined or default set of configuration for each phase and the transition.\newline 
Following is the example or format how the transition config file should be.\newline
E.g. {\it early-x\textgreater mid-y\textgreater late-z\newline}
