% !TEX root = ../../main.tex
\section{Squads}
A squad in this sense is units grouped together that act together, like a squad in real battle. All squads have the Squad class as their base class. 

\paragraph{Creating, deleting, and retrieving squads}
The squad manager class handles all existing squads, new squads are all automatically inserted into the squad manager (via the squad's constructor), the squad must, however, be activated before it executes anything. Since all squads are stored as \texttt{shared\_ptr} they are automatically destroyed when no more references exist. A squad is deleted from the squad manager when it is empty; this can be achieved by either units dying, moving units to another squad, or disband a squad (if the squad is set to disbandable).

Squads can be retrieved using two options, either by the squads id (also automatically set), all squads of a specific type (template function) shown in listing \ref{lst:SquadManager::getSquads()}. The template function will also return all sub-classes to the specified class, meaning \texttt{getSquads<AttackSquad>();} would return both attack squads and drop squads (since drop squad is derived from attack squad).
\begin{lstlisting}[caption={Template function to retrieve squads of the specified type.},label={lst:SquadManager::getSquads()}]
template <typename T>
vector<shared_ptr<T>> SquadManager::getSquads();
\end{lstlisting}
Testing to write

\subsection{The squad base class}


\subsection{Attack squad}

\subsection{Drop squad}

\subsection{Hold squad}

\subsection{Patrol squad}

\subsection{Scout squad}

\subsection{Unit composition}

\subsection{Wait goals}
\label{sec:wait_goals}